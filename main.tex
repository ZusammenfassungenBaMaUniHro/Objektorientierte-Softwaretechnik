\documentclass[11pt, fleqn, a4paper, leqno]{scrartcl} %A4
\usepackage[utf8x]{inputenc} %Eingabe
\usepackage[T1]{fontenc} %Font

\usepackage[ngerman]{babel} %Trennnung
\usepackage{amsmath} %Mathesysmbole
\usepackage{graphicx} %Grafiken
\usepackage{listings} %Programmcode
\usepackage{tikz} %Grafiken malen

\definecolor{mygreen}{rgb}{0,0.6,0}
\definecolor{mygray}{rgb}{0.5,0.5,0.5}
\definecolor{mymauve}{rgb}{0.58,0,0.82}

\lstset{ %
  backgroundcolor=\color{white},   % choose the background color; you must add \usepackage{color} or \usepackage{xcolor}
  basicstyle=\footnotesize,        % the size of the fonts that are used for the code
  breakatwhitespace=false,         % sets if automatic breaks should only happen at whitespace
  breaklines=true,                 % sets automatic line breaking
  captionpos=b,                    % sets the caption-position to bottom
  commentstyle=\color{mygreen},    % comment style
  deletekeywords={...},            % if you want to delete keywords from the given language
  escapeinside={\%*}{*)},          % if you want to add LaTeX within your code
  extendedchars=true,              % lets you use non-ASCII characters; for 8-bits encodings only, does not work with UTF-8
  frame=single,                    % adds a frame around the code
  keepspaces=true,                 % keeps spaces in text, useful for keeping indentation of code (possibly needs columns=flexible)
  keywordstyle=\color{blue},       % keyword style
  language=Octave,                 % the language of the code
  morekeywords={*,...,mean,avg,clone},            % if you want to add more keywords to the set
  numbers=left,                    % where to put the line-numbers; possible values are (none, left, right)
  numbersep=5pt,                   % how far the line-numbers are from the code
  numberstyle=\tiny\color{mygray}, % the style that is used for the line-numbers
  rulecolor=\color{black},         % if not set, the frame-color may be changed on line-breaks within not-black text (e.g. comments (green here))
  showspaces=false,                % show spaces everywhere adding particular underscores; it overrides 'showstringspaces'
  showstringspaces=false,          % underline spaces within strings only
  showtabs=false,                  % show tabs within strings adding particular underscores
  stepnumber=2,                    % the step between two line-numbers. If it's 1, each line will be numbered
  stringstyle=\color{mymauve},     % string literal style
  tabsize=1,                       % sets default tabsize to 2 spaces
  title=\lstname                   % show the filename of files included with \lstinputlisting; also try caption instead of title
}


\title{Zusammenfassung: Objektorientierte Programmierung}
\author{Andreas Ruscheinski}

\begin{document}
\maketitle
\section{Software Patterns}
	\subsection{Einteilung}
		\subsubsection{Anwendungsbereich: Klassenmuster}
			\begin{itemize}
				\item Beziehungen von Klassen zur Compilezeit festsetzen
			\end{itemize}
		\subsubsection{Anwendungsbereich: Objektmuster}
			\begin{itemize}
				\item Beziehungen zwischen Klassen beschreiben Relation von Objketen (Aggregation und Assoziation)
				\item variabel zur Laufzeit
			\end{itemize}
		\subsubsection{Zweck: Erzeugungsmuster}
			\begin{itemize}
				\item Aufgabe: Erzeugung von Objekten
				\item Klassenmuster: nutzen von Vererbung, zum Klasse des zu erzeugenden Objektes zu variieren
				\item Objektmuster: delegieren Objekterzeugung an ein anderes Objekt
				\item Bsp: Anzahl der erzeugten Elemente varieren oder konkrete Typ abhängig von Situation anpassen		
			\end{itemize}
		\subsubsection{Zweck: Strukturmuster}
			\begin{itemize}
				\item Aufgabe: Vereinfachung der Struktur zwischen Klassen
				\item Klassenmuster: nutzen Vererbung, um Schnittstellen und Implementierungen zusammenzuführen
				\item Objektmuster: Objekte zusammenzuführen
				\item Bsp: Vereinfachung von komplexen Beziehungen		
			\end{itemize}
		\subsubsection{Zweck: Verhaltensmuster}
			\begin{itemize}
				\item Aufgabe: Beiflussung mit dem Verhalten von Klassen
				\item Klassenmuster: Vererbung um Verhalten unter den Klassen zu verteilen
				\item Objektmuster: Aggregation anstelle von Vererbung
				\item Bsp: Vereinfachung des Nachrichtenaustausches
			\end{itemize}
	\newpage
	\subsection{Patterns}
		\subsubsection{Observer}
			\begin{itemize}
				\item Kategorie: Verhaltensmuster
				\item Zweck: Bei Änderungen des Zustandes sollen alle abhängigen Objekte benachrictigt und aktualisiert werden 
				\item Akteure:
					\begin{itemize}
						\item Subjekt: kennt Liste von Beobachtern, Schnittstelle für An- und Abmelden von Beobachtern
						\item KonkretesSubjekt: speichert des Zustand und benachrichtigt alle Beobachter bei Änderungen des Zustandens
						\item Beobachter: definiert Aktualisierungssschnittstelle
						\item KonkreterBeobachter: Referenz zum KonkretenSubjekt, speichert den Zustand, implementiert Beobachter
					\end{itemize}

				\item Ablauf:
					\begin{enumerate}
						\item  setState() ändert Zustand des Subjektes
						\item Änderung am Subjekt bewirkt iterative Benachrichtigung der angemeldeten Beobachter
						\item Bei Benachrichtung des Beobachters holt dieser den neuen Zustand mit getState()
					\end{enumerate}
				\item Beispiel: Änderungen an der Börse(Konkretes Subjekt) sollen alle Broker(Konkreter Beobachter) benachrichtigt werden
			\end{itemize}
			\includegraphics[scale=0.5]{images/observer.png}
		\newpage
		\subsubsection{Prototype}
			\begin{itemize}
				\item Kategorie: Erzeugungsmuster
				\item Zweck: Instanz eines Objektes wird zur Erzeugung weiterer Objekte genutzt, welche dabei Kopien der Instanz sind
				\item Akteure: 
					\begin{itemize}
						\item Prototype: Schnittstelle welche das Klonen von Objekten zusichert
						\item KonkreterPrototype: Implementiert die Schnittstelle
					\end{itemize}
				\item Ablauf:
					\begin{enumerate}
						\item Klient ruft clone() auf dem Prototype auf
						\item Klient bekommt KonkreterPrototype1 oder KonkreterPrototype2, abhängig von dem hinterlegten Prototype
					\end{enumerate}
				\item Beispiel: 
			\end{itemize}
			\begin{lstlisting}[frame=single,language=Java]
				data = getData();
				//Analyse der Daten durch verschiedene Functionen
				avgData = avg(data.clone());
				meanData = mean(data.clone());
				//data hat sich nicht veraendert
			\end{lstlisting}
			\includegraphics[scale=0.6]{images/prototype.png}
		\newpage		
		\subsubsection{Command}
			\begin{itemize}
				\item Kategorie: Verhaltensmuster
				\item Zweck: Für ein interaktives System soll ein Mehrstufiges, nicht limitiertes rückgängig Machen und Wiederholen von Aktionen möglich seins
				\item Akteure: 
					\begin{itemize}
						\item Klient: erzeugt KonkreterBefehl mit Verweis zum Empfänger, gibt Aufrufer eine Referenz auf den KonkretenBefehl
						\item Empfänger: KonkreterBefehl ruft Methoden des Empfängers um seine Aktion auszuführen
						\item Aufrufer: Besitzt Verweise auf Befehle, fordert Befehle  bei Bedarf auf
						\item Befehl: definiert Schnittstelle zum Ausführen eines Befehls
						\item KonkreterBefehl: implementiert Schnittstelle, speichert Referenz zum Empfänger und für die Ausführung nötigen Zustand
					\end{itemize}
				\item Ablauf:
					\begin{enumerate}
						\item Client führt Befehl aus
						\item Befehl kennt Empfänger und ruft execute() auf Empfänger auf
						\item Implementierung von Undo und Redo über Speicherung der Befehle im Aufrufer
					\end{enumerate}
				\item Beispiel: zeilenweises Editieren mit einem Texteditor (Löschen, Ersetzen, Einfügen) und diese Aktionen rückgängigmachen
			\end{itemize}
			\includegraphics[scale=0.8]{images/commando.png}
			\newpage
		\subsubsection{Iterator}
			\begin{itemize}
				\item Kategorie: Verhaltensmuster
				\item Zweck: Bereitstellung einer sequentiellen Reihenfolge von zusammengesetzte Objekte (Liste, Bäume, etc) mit Abstraktion von der Datenstruktur
				\item Akteure: 
					\begin{itemize}
						\item Iterator: definiert Schnittstelle für den Besuch von Elementen
						\item KonkreterIterator: implementiert Schnittstelle für den Besuch von Elementen, achtet auf die aktuelle Position in der Datenstruktur
						\item Aggregat: definiert die Schnittstelle zur Erzeugung eines Iterators
						\item KonkretesAggregat: implemtiert Schnittstelle zur Erzeuging eines Iterators
					\end{itemize}
				\item Ablauf:
					\begin{enumerate}
						\item KonrketerIterator implemtieren mit gewünschter Logik
						\item KonkretesAggregat implementierenf
						\item Arbeit mit dem Iterator
					\end{enumerate}
				\item Beispiel: Iterator über eine Liste mit Filterausdruck
			\end{itemize}
			\includegraphics[scale=0.8]{images/iterator.png}
			\newpage
		\subsubsection{Mediator}
			\begin{itemize}
				\item Kategorie: Verhaltensmuster
				\item Zweck: Steuerung der Kommuinikation zwischen Objekten über einen gemeinsamen Vermittler
				\item Akteure: 
					\begin{itemize}
						\item Vermittler: definiert Schnittstelle für die Kommunikation mit den Kollegen
						\item KonkreterVermittler: implementiert die Vermittler Schnittstelle für das kooperative Verhalten durch die Koordination der beteiligten Kollegen, kennt und verwaltet Kollegen-Objekte
						\item Kollegen-Klassen: jeder Kollege kennt sein Vermittler, kommunikation mit dem Vermittler anstatt mit dem anderen Objekt Kontakt aufzunehmen
					\end{itemize}
				\item Ablauf:
					\begin{enumerate}
						\item Kollege benachrichtigt Mediator über eine Änderung
						\item Mediator benachricht alle anderen Beteiligten
					\end{enumerate}
				\item Hinweis: sehr änhnlich zum Observer, Unterschied: Observer 1:n Kommunikaton; Mediator: n:m Kommunikation
				\item Beispiel: Chatroom Implementierung, Kollegen sind Benutzer (Bots oder Menschen), Vermittler ist ein Zentraler Server der alle Kollegen bei Nachrichten benachrichtigt			
			\end{itemize}
			\includegraphics{images/mediator.png}
			\newpage
		\subsubsection{Abstract Factory}
			\begin{itemize}
				\item Kategorie: Erzeugungsmuster
				\item Zweck: Bereitstellung eines Interfaces zur Erzeugung von Objekt-Familien zur Laufzeit, ohne deren konkrete Klassen festzulegen
				\item Akteure: 
					\begin{itemize}
						\item AbstrakteFabrik: definiert Schnittstelle für die Erzeugung AbstrakterProdukte einer Produktfamilie
						\item KonkreteFabrik: erzeugt KonkreteProdukte durch implementierung der AbstrakteFabrik Schnittstelle
						\item AbstraktesProdukt: definiert Schnittstelle für eine Produktart
						\item KonkretesProdukt: implementiert die AbstraktesProdukt Schnittstelle
						\item Klient: Nutzt die Schnittstelle der AbstractenFabrik und AbstrakterProdukte für weiteren Arbeitsverlauf 
					\end{itemize}
				\item Ablauf:
					\begin{enumerate}
						\item Definition der Schnittstellen
						\item Implementierung von KonkratenProdukten
						\item Implementierung von KonkretenFabriken
						\item Klient arbeitet auf den Abstrakten Schnittstellen 
					\end{enumerate}
				\item Beispiel: GUI-Programmierung, AbstraktesProdukt: Button, KonkretesProdukt: LinuxButton, WindowsButton, AbstrakteFabrik: GUI-Dialog, KonkrateFabrik: Linux-GUI-Dialog(nutzt LinuxButton) usw...
			\end{itemize}
			\includegraphics[scale=0.5]{images/abstract-factory.png}
			\newpage
		\subsubsection{Builder}
			\begin{itemize}
				\item Kategorie: Erzeugungsmuster
				\item Zweck: Trennung der Konstruktion komplexer Objekte von ihrer Repräsentation. Dadruch kann der gleiche Prozesss verschiedene Repräsentationen erzeugen.
				\item Akteure: 
					\begin{itemize}
						\item
					\end{itemize}
				\item Ablauf:
					\begin{enumerate}
						\item
					\end{enumerate}
				\item Beispiel:
			
			\end{itemize}
\end{document}
				\item Kategorie:
				\item Zweck:
				\item Akteure: 
					\begin{itemize}
						\item
					\end{itemize}
				\item Ablauf:
					\begin{enumerate}
						\item
					\end{enumerate}
				\item Beispiel:
